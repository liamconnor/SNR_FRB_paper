\documentclass[useAMS,usenatbib]{mn2e}


%define general packages
%\usepackage{epsfig}
\usepackage{amsmath}
%\usepackage{natbib}
%\usepackage{epstopdf}

\usepackage{epsfig}
\usepackage{epstopdf}
\usepackage{lscape} % Allows landscape environment to be used
\usepackage{natbib}
\usepackage{tabularx}
\usepackage{multirow}
\usepackage{amssymb}
\def\gtrsim{\mathrel{\hbox{\rlap{\hbox{\lower4pt\hbox{$\sim$}}}\hbox{$>$}}}}



\title{Non-Cosmological FRB's from Young Supernova Remnant Pulsars}
\author[Connor et al.]{
Liam Connor$^{1,2,3}$\thanks{E-mail:\ connor@astro.utoronto.ca}
Jonathan Sievers $^{4}$\thanks{E-mail:\ jonathan.sievers@gmail.com}
Ue-Li Pen $^{1, 5,6}$\thanks{E-mail:\ pen@cita.utoronto.ca}
\\
$^1$ Canadian Institute for Theoretical Astrophysics, University of Toronto, M5S 3H8 Ontario, Canada
\\
$^2$ Department of Astronomy and Astrophysics, University of Toronto, 
M5S 3H8 Ontario, Canada
\\
$^3$ Dunlap Institute for Astronomy and Astrophysics, University of Toronto,
Toronto, ON M5S 3H4, Canada
\\
$^4$ Astrophysics and Cosmology Research Unit, University of KwaZulu-Natal, Durban, South Africa
\\
$^5$ Canadian Institute for Advanced Research, Program in Cosmology
and Gravitation
\\
$^6$ Perimeter Institute for Theoretical Physics, 31 Caroline St. N., Waterloo, ON, N2L 2Y5, Canada
}


% Usually omit these for ApJ or MNRAS style files:
%\tableofcontents
%
%\listoffigures
%
%\listoftables

\begin{document}
\date{\today}
\pagerange{\pageref{firstpage}--\pageref{lastpage}} 
\pubyear{2015}
\maketitle
\label{firstpage}


\begin{abstract}

We  propose a new extragalactic but non-cosmological explanation for FRB's based on
very young pulsars in supernova remnants. Within a few hundred years of a 
core-collapse supernova the ejecta 
is confined within $\sim$1 pc, providing a high enough column density of free electrons 
for the observed 500-1500 pc/cm$^3$. By extrapolating a Crab-like pulsar to 
its infancy in an environment like that of SN 1987A, 
we hypothesize such an object could emit supergiant pulses sporadically which 
would be bright enough to see at a few hundred megaparsecs. In this scenario Faraday
rotation at the source gives RM's much larger than the expected
cosmological contribution.  If the emission were pulsar-like, then the polarization  % Reword, don't like this
vector could swing over the duration of the burst, which is not expected from 
non-rotating objects.
In this model, the scattering,
large DM, and commensurate RM all come from one place 
 which is not the case for the cosmological
interpretation.  The model also provides
testable predictions of the flux distribution and repeat rate of fast radio bursts, and could be further
verified by spatial coincidence with optical supernovae of the past several decades. 
\end{abstract}
\begin{keywords}
FRB, supernova remnants, pulsar, giant pulse
\end{keywords}
%\begin{keywords}
%FRB, magnetars, galactic center, giant pulse
%\end{keywords}

%\footnote{Canadian Institute for Theoretical Astrophysics, University of Toronto, M5S 3H8 Ontario, Canada} \\
%\footnote{Department of Astronomy and Astrophysics, University of Toronto, 
%M5S 3H8 Ontario, Canada}
%\footnote{Dunlap Institute for Astronomy and Astrophysics, University of Toronto,
%Toronto, ON M5S 3H4, Canada}
%\footnote{Astrophysics and Cosmology Research Unit, University of KwaZulu-Natal, Durban, South Africa}
%\footnote{Canadian Institute for Advanced Research, Program in Cosmology
%and Gravitation}\\

\newcommand{\be}{\begin{eqnarray}}
\newcommand{\ee}{\end{eqnarray}}
\newcommand{\beq}{\begin{equation}}
\newcommand{\eeq}{\end{equation}}

\section{Introduction}
%\\
The mystery of fast radio bursts (FRB's) has garnered
substantial interest from the radio community.
High-energy astrophysicists have tried to model their burst source, 
observers would like to measure a large population of them, and cosmologists
hope to use them as a probe of the IGM. However their relative scarcity 
(only $\sim$ dozen have been observed so far) and their apparent 
transient nature have meant we still do not know their position on the sky
to better than a few arcminutes, and their radial position could be anything
from terrestrial to cosmological.

These objects are
highly dispersed, with DM's ($\sim 500$-1200 pc/cm$^3$) far exceeding
the expected contribution from our own galaxy's ISM and leading to the
interpretation that FRB's are cosmological \citep{2007Sci...318..777L, 2013Sci...341...53T}. 
Various emission mechanisms have been proposed 
at a wide range of source locations, 
including merging white dwarfs \citep{2012ApJ...760...64M}
and neutron stars \citep{2013PASJ...65L..12T},
blitzars \citep{2014A&A...562A.137F}, 
magnetars \citep{2015arXiv150101341P, 2014MNRAS.442L...9L}, 
and flaring galactic stars \citep{2014MNRAS.439L..46L}.
Though presently there are more theoretical models for FRB's than actual 
sources discovered, constraints on such theories are rapidly emerging. 
This is due to recent polarization data, 
multifrequency coverage, and their being observed by several telescopes
at various locations on the sky \citep{2014ApJ...780L...2B, 2014arXiv1412.0342P}. 

On top of event rates ($\sim$10$^4$ per day per sky) 
and high DM's, explanations of FRB's must now
account for temporal scattering, and polarization states.  They should
predict or explain Faraday rotation and time dependence of linear polarization.
The rotation measure of our galaxy has been mapped, and the
intergalactic rotation measure is constrained to be less than 7 rad/m$^2$
\citep{2015A&A...575A.118O}. 
The observed temporal scattering is problematic for a IGM intepretation, due 
to the unrealistically small length scales required in the IGM 
for 1 ms scattering \citep{2014ApJ...785L..26L}. 

%\\
In this letter we propose a new non-cosmological but extragalactic
solution to the FRB problem: giant pulses from newly formed pulsars in 
supernova remnants (SNR's). The dense ionized environment of the SNR
can provide 500-2000 pc/cm$^3$ of dispersion if the pulses are observed 
within $\sim100$ years of the core-collapse supernova. In our picture the 
large DM and scattering all come from the same place, and generically
accounts for substantial Faraday Rotation and polarization angle
swings\footnote{\label{disclosure}The authors disclose access to unpublished data prior
  to making these predictions}.
These are not expected in a cosmological interpretation of the DM.


% -- Scattering arguments, same as in the magnetar paper.

\section{Supernova Remants}
Of order 10$^{51}$ ergs of kinetic energy is released during a supernova, a 
fraction of which is converted into thermal 
energy after shock heating of the 
ejecta plasma. Though the shock-heated ejacta atoms 
are fully ionized after the explosion, the density is high enough that
ionized atoms can soon recombine.
This phase of low-ionization comes to an end when the remnant expands 
into the surrounding ISM, causing a reverse shock wave that reionizes the ejecta.
%\\
Though this is the basic narrative, observations \citep{2014ApJ...796...82Z} 
as well as simulations \citep{2014ApJ...794..174P}
of SN 1987a have shown the morphological and ionization properties of SNR's
in the decades and centuries after the explosion are nuanced and 
difficult to model.
That said, in general the expanding nebula left behind 
should be able to provide enough free electrons
along the line-of-sight for unusually large dispersion measures. If we 
assume a toy model in which a spherical shell expands at $v_{ej}$, 
then the radius R(t) $\approx v_{ej} t$. Therefore the DM we expect can be 
calculated as,

\begin{equation}
\begin{centre}
\textup{DM} \approx  \frac{\textup{x}_e \textup{M}_{ej}}{m_p \frac{4\pi}{3} v_{ej}^2 t^2}
\end{centre}
\end{equation}

\noindent where x$_e$ is the ionization fraction, 
M$_{ej}$ is the ejecta mass, and $m_p$ 
is the mass of a proton. Assuming $\sim$10 M$_{\odot}$ of material 
is ejected at $v_{ej}\sim 3-8\times10^3$ km/s and an ionization fraction of 
$\sim 20\% $, the dispersion measure goes from several 
thousand pc/cm$^3$ immediately
after the reverse-shock ionization, to several hundred pc/cm$^3$ after 50-100 years.
A similar treatment by \cite{2014ApJ...796...82Z} found that a possible pulsar in SNR 
1987a could have DM's between 100-6000 pc/cm$^3$, after $\sim 25$ years.

%\\
Another potentially important feature of the SNR environment is its magnetic
field. 
The exact magnitude of any detection Faraday rotation has implications for the possible source location. For
instance in the circumnuclear picture, one would expect rotation measures 
$\sim10^{3-5}$ rad/m$^2$, similar to that of the Milky Way's
galactic center magnetar J1745-29. In the cosmological scenario, if the Faraday 
rotation
came from the same place as the DM - namely the intergalactic medium -
then we would only expect a few rad/m$^2$ of RM \cite{2015A&A...575A.118O}. 

The Faraday effect rotates the polarization vector
by an angle $\phi = $RM$\, \lambda^2$, where

\begin{equation}
\textup{RM} = \frac{e^3}{2\pi m^2 c^4} \int_0^{L} n_e(l) B_\parallel (l) dl
\end{equation}

We can therefore make a rough estimate of the rotation measure of a remnant 
pulsar with dispersion measure DM. Using 
\cite{2014ira..book.....B} we get,

\begin{equation}
\textup{RM} \approx 0.81\,\textup{rad} / m^2 \, \times \frac{\left < B_{\parallel} \right >}
{1 \mu G} \cdot \frac{\textup{DM}}{1 pc/cm^3} 
\end{equation}

Though there is a large uncertainty in evolution of the magnetic field strength and added
uncertainty in $\langle B_{\parallel} \rangle$ given $B_{\parallel}$ is not necessarily positive, 
typical values in our galaxy are 0.2 - 1$\mu G$. For instance the Crab and Vela have 
$ \sim 0.92 \mu G$ and $\sim 0.56 \mu G$, respectively. 
This gives RM's between $\sim 80-1200$
rad/m$^2$ for a SNR pulsar with FRB-like DM's.
%and this is consistent with the observed Faraday rotated burst.

\subsection{Event Rates}

The daily FRB rate has been estimated at $3.3^{+5.0}_{-2.5}\times10^3$ sky$^{-1}$ 
\citep{2015arXiv150500834R}. If we start from the local core-collapse supernova
event rate, $\Gamma_{CC}$, and include objects out to some distance $d_{max}$,
we expect the following daily FRB rate, 

\begin{equation}
\Gamma_{FRB} \sim  \frac{4}{3} \pi d_{max}^3 \times \Gamma_{CC} \times
 \eta \, \tau_{ion} \gamma_{GP}
\end{equation}

\noindent where $\tau_{ion}$ is the window in years when the SNR is sufficiently
dense and ionized to provide the observed DM's and $\gamma_{GP}$
is the daily rate of giant pulses brighter than $\sim 500$ mJy, and $\eta$
is the number of core-collapse supernovae that leave behind a visible pulsar. 
From \cite{2014ApJ...792..135T} we know  
$\Gamma_{CC}\sim3 \times 10^{-4}$ day$^{-1}$ ($h^{-1}$Mpc)$^{-3}$,
so if we take $d_{max}$ to be 100 $h^{-1}$Mpc and $\tau_{ion}\sim100$ years,
we require one giant pulse every 10-20 days, assuming one fifth of this SNe population
leaves behind a visible pulsar.
In figure \ref{FIG-RATE} 
we show the event rate as a function of distance, varying two parameters: the 
effective high-DM window and the rate of giant pulses. 

\begin{figure}
  \centering
   \includegraphics[width=0.5\textwidth]{FRB_SNR_rate.png}
   \caption{Daily FRB rate per sky based on local core-collapse supernova 
   event rate, plotted against distance.
   We assume early in the pulsar's life there is a window, either 
   25, 100, or 500 years when the SNR can provide a large enough electron 
   column density to explain the high DM's of the observed bursts. We also
   include a rate of giant pulses of either one per day or one per hundred
   days. We have assumed 20$\%$ of core-collapse supernovae leave behind
   a visible pulsar.
   The horizontal black lines are the $99\%$ confidence bounds for the FRB rate
   found by \cite{2015arXiv150500834R}.}
   \label{FIG-RATE}
\end{figure}

From this figure we can see even in our most conservative estimate, when
the SNR only has a 25 year window and emits giant pulses once every
100 days, the volume necessary for the highest daily FRB rate is still non-cosmological.
By this we mean the DM contribution from the IGM is less than $\sim 200$ pc/cm$^3$.
If the SNR FRB's are within a hundred $h^{-1}$Mpc then DM$_{IGM}$ is less than 
$\sim 10 \%$  of the total dispersion of a typical burst.

If FRB's really are giant pulses then they 
should repeat stochastically, and while none of the radio follow-ups for
observed sources has seen an FRB repeat, this could be because they
have not observed for long enough. We discuss this further in section 
\ref{sec-predictions}.

\subsection{Young SNR Pulsars}

About a dozen pulsars in our galaxy are known to emit extremely energetic,
short duration radio pulses which can be many orders of magnitude 
brighter than the pulsar's regular emission. Some of these objects exhibit 
a rare tail of \textit{supergiant} pulses, whose brightness temperatures 
exceed the Planck temperature, $\gtrsim10^{32}$ K \citep{2004ApJ...612..375C}.
Indeed the largest
known brightness temperature in the universe came from a giant pulse from the Crab,
 with T$_b\sim2\times10^{41}$ K \citep{2014ApJ...792..135T}. Though there
is only $\sim100$ hours of published giant pulse data from the Crab, it is known
that the supergiant pulse tail does not obey the standard power-law fall off
in amplitude \citep{2012ApJ...760...64M}.
% If an instrument with very little forward gain and no real-time
%RFI cut, like a single dipole, were to monitor the Crab from several $\sim$ months
%we would have a better idea of the statistics and frequency of such events. 

Given the relatively high frequency of core-collapse supernovae 
in the local universe, the young 
rapidly rotating pulsars such events leave behind could emit giant 
pulses bright enough to be observed at hundreds of megaparsecs. 
These supergiant pulses would require $10^{36 - 37}$ ergs of output,
assuming and observed flux density of 0.3-5 Jy and $\sim500$ 
MHz of bandwidth over 1 ms. Though this is $\sim$ billions of times
brighter than an average pulse, it is negligible compared to a 
pulsar's total rotational energy, E$_{rot} \sim 10^{49-50}$ ergs. We
also point out that given its relative proximity, this model requires
a couple orders of magnitude less energy than cosmological FRB's,
located beyond a Gpc. 

% Please add the following required packages to your document preamble:
% \usepackage{multirow}

\begin{table*}
\label{TAB-1}
\begin{tabularx}{1.08\textwidth}{@{\extracolsep{\fill}}|lccccccc|}
\hline
\multicolumn{1}{|c}{\textbf{Location}}                                                                                            & \textbf{Model}                                              & \textbf{\begin{tabular}[c]{@{}c@{}}$\mu$sec\\ scintillation\end{tabular}} & \textbf{\begin{tabular}[c]{@{}c@{}}Faraday \\ rotation\end{tabular}} & \textbf{$\mathbf{\frac{dlnN_{FRB}}{dlnS_{\nu}}}$}                                      & \multicolumn{1}{l}{\textbf{Counterpart}}                                    & \textbf{\begin{tabular}[c]{@{}c@{}}DM range\\ (pc/cm$^3$)\end{tabular}} & \textbf{\begin{tabular}[c]{@{}c@{}}Pol angle \\ swing\end{tabular}} \\ \hline
\multicolumn{1}{|l|}{\multirow{4}{*}{\begin{tabular}[c]{@{}l@{}}Cosmological \\ ($\gtrsim 1h^{-1}$Gpc)\end{tabular}}}             & Blitzars                                                    & No                                                                        & $\lesssim 7$ rad/m$^2$                                               & ?                                                                                      & \begin{tabular}[c]{@{}c@{}}gravitational \\ waves\end{tabular}              & 300-2500                                                                & ?                                                                   \\
\multicolumn{1}{|l|}{}                                                                                                            & Merging COs                                                 & No                                                                        & $\lesssim 7$ rad/m$^2$                                               & ?                                                                                      & \begin{tabular}[c]{@{}c@{}}Type Ia SNe,\\  X-ray, $\gamma$-ray\end{tabular} & 300-2500                                                                & No                                                                  \\
\multicolumn{1}{|l|}{}                                                                                                            & Primordial BHs                                              & No                                                                        & $\lesssim 7$ rad/m$^2$                                               & $\leq$-3/2                                                                             & $\sim$TeV                                                                   & 300-2500                                                                & No                                                                  \\
\multicolumn{1}{|l|}{}                                                                                                            & Magnetar flare                                              & No                                                                        & $\lesssim 7$ rad/m$^2$                                               & ?                                                                                      & \begin{tabular}[c]{@{}c@{}}$\sim$ms TeV \\ burst\end{tabular}               & 300-2500                                                                & $\checkmark$                                                        \\ \cline{1-1}
\multicolumn{1}{|l|}{\multirow{3}{*}{\begin{tabular}[c]{@{}l@{}}Extragalactic, local \\ ($\lesssim$200$h^{-1}$Mpc)\end{tabular}}} & Edge-on disk                                                & $\checkmark$                                                              & 50-500 rad/m$^2$                                                     & -3/2                                                                                   & ?                                                                           & 10-2000                                                                 & ?                                                                   \\
\multicolumn{1}{|l|}{}                                                                                                            & \begin{tabular}[c]{@{}c@{}}Nuclear \\ magnetar\end{tabular} & $\checkmark$                                                              & 10$^{3-5}$ rad/m$^2$                                                 & -3/2                                                                                   & None                                                                        & 10-3000                                                                 & $\checkmark$                                                        \\
\multicolumn{1}{|l|}{}                                                                                                            & SNR pulsar                                                  & $\checkmark$                                                              & 20-$10^3$ rad/m$^2$                                                  & -3/2                                                                                   & \begin{tabular}[c]{@{}c@{}}Archival CC \\ SNe\end{tabular}                  & 10$^2$-10$^4$                                                           & $\checkmark$                                                        \\ \cline{1-1}
\multicolumn{1}{|l|}{Galactic ($\lesssim 100$kpc)}                                                                                & Flaring MS stars                                            & $\checkmark$                                                              & -RM$_{gal}$                                                          & -3/2                                                                                   & \begin{tabular}[c]{@{}c@{}}Main sequence \\ star\end{tabular}               & $\gtrsim$ 300                                                           & No                                                                  \\ \cline{1-1}
\multicolumn{1}{|l|}{Terrestrial ($\lesssim 10^4$km)}                                                                             & RFI                                                         & No                                                                        & -RM$_{gal}$                                                          & $\left\{\begin{matrix}-1/2 \,\, if \,\, 2D \\ -3/2 \,\, if \,\, 3D\end{matrix}\right.$ & None                                                                        & ?                                                                       & No                                                                  \\ \hline
\end{tabularx}
\caption{This table summarizes a number of FRB models by classifying them as cosmological, 
extragalactic but non-cosmological, galactic, and terrestrial. 
The seven columns are potential observables of FRB's and each
 row gives their consequence for a given model 
 (Blitzars \citep{2014A&A...562A.137F}, compact object mergers \citep{2012ApJ...760...64M, 2013PASJ...65L..12T},
 exploding primordial blackholes \citep{2014PhRvD..90l7503B}, bursts from magnetars \citep{2014MNRAS.442L...9L}, edge-on disk galaxies \citep{2015arXiv150400200X}, circumnuclear magnetars \citep{2015arXiv150101341P}, 
 supernova remnant pulsars, stellar flares \citep{2014MNRAS.439L..46L}, and terrestrial RFI 
 \citep{2015arXiv150305245H}.)
 . Since scintillation
only affects unresolved images, cosmological sources that presumably have already been scatttered
will not scintillate in our galaxy, while non-cosmological sources will. For Faraday rotation we assume 
the RM comes from the same place as the DM, e.g. the IGM for cosmological sources. Even though
all models have to explain the observed 500-1500 pc/cm$^3$, some models predict a wider 
range of DM. For instance, in the circumnuclear magnetar or edge-on disk disk scenarios there 
ought to be bursts at relatively low DM that simply have not been identified as FRB's. In our supernova 
remnant model DM's should be very large early in the pulsar's life, though this window is short and 
therefore such high DM bursts would be rare.}
\end{table*}

The polarization properties of giant pulses are also consistent with
observed those of FRB's. Giant pulses are known to be highly polarized, 
switching between strong Stokes V and purely linearly polarized states. 
The only published FRB with full-pol information was FRB 140514 was 
found to have $\sim20\%$ circular polarization and no detectable 
linear polarization \citep{2014arXiv1412.0342P}.  We expect some FRB's
to be linearly polarized, and potentially exhibit polarization angle
swings over the burst duration.

\section{Predictions}
\label{sec-predictions}

In table \ref{TAB-1} we try and summarize the observational consequences
of ours and several other models as best we can. As one might expect,
the most striking differences in predictions has to do with the distance of FRB's,
for example the cosmological FRB models largely differ in their expected 
counterpart. 

The young SNR pulsar model makes several predictions that will
be addressed with more data, particularly with full polarization 
observations and large field-of-view surveys. 
The latter will provide a large sample of FRB's whose flux and DM statistics
 can give us information about their location. Since in the SNR FRB picture
most of the DM is intrinsic, the sources do not need to be at cosmological 
distances. This means the flux distribution is given by a Euclidean universe
that is only weakly dependent on DM, $N(>S) \propto S^{-3/2}$. Surveys
like CHIME \citep{2014SPIE.9145E..22B}, 
UTMOST\footnote{http://www.caastro.org/news/2014-utmost},
 or HYRAX could observe as many as $\sim10^{3-4}$
per year, which would allow for detailed population statistics.
%\\

Since we have proposed that FRB's come from young pulsars 
in supernova remnants, it is possible that the corresponding 
supernova was observed in recent decades in the optical. If the pulsars
were younger than $\sim$60 years old they could be localized with VLBI
measurements and matched against catalogued type II supernovae. 
%\\

We also point out that while FRB's seem not to repeat regularly, 
it is not known that they never repeat. Though the statistics 
of giant pulses from local pulsars are mostly Poisson \citep{1999ApJ...517..460S},
it is possible that 
the supergiant pulses we require from very young SNR pulsars are not. If their 
statistics were of a Poisson process then there are already limits on the repeat rate, 
given the $\sim100$ hours of follow up, however if their statistics were more like
earthquakes, the brightest pulses could burst intermittently and turn off
for extended periods. It is possible that FRB's could repeat every 5-500 days. 
If they were to repeat,
it is possible that their DM's, RM's, and scattering properties could 
change noticeably on months/years timescales. Unlike standard pulsars 
whose RM's and DM's are constant to a couple decimal places, young 
SNR pulsars like the Crab and Vela have shown significant - and sometimes
correlated - variation in such properties \citep{1988A&A...202..166R, 2008A&A...483...13K}.
. We also predict that such repeated 
bursts could have vastly different polarization states, similar to the giant 
pulses from pulsars in our own galaxy. 
Depending on the relationship between the giant pulse rate and SNR
age and environment, there may exist a short window in the pulsar's life when 
DM's are larger than could be achieved in the IGM. It would be therefore 
possible, albeit rare, that an FRB have a dispersion
 measure of $\sim10^4$ pc/cm$^3$.
 
Another interesting path for studying extragalactic radio bursts, 
cosmological or otherwise, is scintillation. Only objects of small angular
size scintillate, which is why stars twinkle and planets do not: turbulent cells
in the ionosphere can resolve planets but not stars. The same is true for extragalactic
objects scintillating in the milky way, where objects larger than $\sim10^{-7}$ 
arc seconds do not scintillate at $\sim$GHz.
This is why so few quasars scintillate \citep{2002Natur.415...57D}. 

Using \cite{1986isra.book.....T} 
we can estimate the angular size of an extragalactic object,

\begin{equation}
\label{eqn-scint}
\theta \approx \left ( \frac{2 c \tau \, (R_{obj} - R_{sn})}{R_{sn}R_{obj}} \right )^{-1/2}
\end{equation}

\noindent where $R_{obj}$ is the distance to the source, $R_{sn}$ is the distance 
to the screen, and $\tau$ is the scattering timescale. For of FRB's we 
take $\tau$ to be $\sim10$ ms. In the cosmological case, if the $\ms$ scattering
were from an extended galactic disk along the line of sight  (see \cite{2014ApJ...780L..33M})
halfway between
us and the source, then the angular broadening
of an object at 2 Gpc is $\sim150$ microarcseconds. If the screen were within 
1 kpc of the same object then the broadening is $\sim80$ nanoarcseconds. 
Therefore scintillation from our own galaxy should only occur for cosmological 
FRB's whose millisecond scattering is close to the source. For an SNR FRB the 
screen would have to be within a few hundred parsecs of the object, which we 
generically expect. We include this feature in table \ref{TAB-1} where each 
column is estimated based on the medium that is causing the high dispersion measure, 
e.g. the IGM for cosmological models.


%\section{Applications}


\section{Conclusions}
Evidence is piling on suggesting FRB's are not only extraterrestrial
but extragalactic. Though the simplest interpretation of their high DM's 
is a cosmological one, we find this model less compelling in the light of 
scattering measurements and potential Faraday rotation$^\ref{disclosure}$
  and in this letter we offer a 
more nearby solution. 
We have gone through
a model in which fast radio bursts are really supergiant pulses from 
extragalactic supernova remnant pulsars, within a couple hundred megaparsecs. 
The SNR environment is sufficiently
dense and ionized to provide DM's $\gtrsim 500$pc/cm$^3$ as well as 
RM's $\gtrsim 50$ rad/m$^2$, only the first of which could be replicated by the IGM. 

The nebula could also provide $\sim$ms scattering at 1 GHz, as has been 
observed in galactic supernova remnant pulsars. 
That makes this picture self-contained in the sense that
the young remnant environment can account for the dispersion 
and scattering measure seen in FRB's.  It predicts a higher Faraday
rotation than the IGM, but not as high as galactic centers.  The
repetition rate is related to the distance, and could be from days to
years.  
By extrapolating Crab-like giant pulses back to the pulsar's first century or so,
we have proposed that such objects can emit extremely energetic bursts sporadically. 
If these are similar to giant pulses from galactic pulsars, they could be highly polarized, 
either linearly or circularly, and if they were to repeat their polarization state may 
change drastically. Given the object's rotating nature, polarization
 angles would be likely to swing during the pulse (we
remind the reader of the previous disclosure).
\\




\section{Acknowledgements}

We thank NSERC for support. We also thank Bryan Gaensler, Nlels Oppermann, 
Giovanna Zanardo, and Chris Matzner for helpful discussions. 

\newcommand{\araa}{ARA\&A}   % Annual Review of Astronomy and Astrophys.
\newcommand{\afz}{Afz}       % Astrofizica
\newcommand{\aj}{AJ}         % Astronomical Journal
\newcommand{\azh}{AZh}       % Astronomicekij Zhurnal
\newcommand{\aaa}{A\&A}      % Astronomy and Astrophysics
\newcommand{\aas}{A\&AS}     % Astronomy and Astrophys. Supplement Series
\newcommand{\aar}{A\&AR}     % Astronomy and Astrophysics Review
\newcommand{\apj}{ApJ}       % Astrophysical Journal
\newcommand{\apjs}{ApJS}     % Astrophysical Journal Supplement Series
\newcommand{\apjl}{ApJ}      % Astrophysical Journal Letters
\newcommand{\apss}{Ap\&SS}   % Astrophysics and Space Science
\newcommand{\baas}{BAAS}     % Bulletin of the American Astron. Society
\newcommand{\jaa}{JA\&A}     % Journal of Astronomy and Astrophysics
\newcommand{\mnras}{MNRAS}   % Monthly Notices of the Roy. Astron. Society
\newcommand{\nat}{Nat}       % Nature
\newcommand{\pasj}{PASJ}     % Publ. of the Astron. Society of Japan
\newcommand{\pasp}{PASP}     % Publ. of the Astron. Society of the Pacific
\newcommand{\paspc}{PASPC}   % Publ. Astron. Soc. Pacific Conf. Proc.
\newcommand{\qjras}{QJRAS}   % Quart. Journal of the Royal Astron. Society
\newcommand{\sci}{Sci}       % Science
\newcommand{\solphys}{Solar Physics}       % 
\newcommand{\sova}{SvA}      % Soviet Astronomy
\newcommand{\aap}{A\&A}
\newcommand\jcap{{J. Cosmology Astropart. Phys.}}%
\newcommand{\prd}{Phys. Rev. D}

%\bibliography{frb}
\bibliography{SNRFRB_April30_MNRAS}
\bibliographystyle{mn2e}


\label{lastpage}


\end{document}


